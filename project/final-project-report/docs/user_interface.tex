\section{Brugergrænseflade}

Til anvendelse af systemet skal en mobiltelefon bruges til at sende sms. De mulige kommandoer er vist i tabel~\ref{tab:commands}. Kan brugeren ikke huske de mulige kommandoer, kan vedkommende sende et 'C' og derved få en liste over mulige kommandoer tilsendt. Hvis GSM modulet modtager en ukendt kommando vil brugeren modtage en besked som gør opmærksom på dette.

\vskip0.5cm

\begin{table}[h]
	\centering
	\begin{tabular}{|c|l|}
		\hline
		\rowcolor[HTML]{EFEFEF} 
		\multicolumn{1}{|l|}{\cellcolor[HTML]{EFEFEF}\textbf{SMS}} & \textbf{Sensor}	\\ 	\hline
		T & Temperatur	\\ 	\hline
		A & Altitude	\\ 	\hline
		P & Lufttryk	\\ 	\hline
		C & Anerkendte kommandoer	\\ 	\hline		
	\end{tabular}
	\caption{Kommandooversigt.}
	\label{tab:commands}
\end{table}


\subsection{LED indikation}

Til at indikere hvor systemet er kommet til i programforløbet er der udviklet et LED driver, der bruges til at fortælle brugeren hvilke handlinger der udføres. 
De forskellige indikations mønstre er beskrevet i afsnittet om System Diagnosticering. 