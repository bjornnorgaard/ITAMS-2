\subsection{MC35 GSM}

Dette GSM modem er valgt fordi skolen havde det på lager og det var let tilgængeligt.
Under implementeringen af en driver har der været et stort fokus på at overholde den timings, da den ellers ikke vil fungere konsistent.\\

Når GSM modulet modtager en SMS, bliver den gemt i et array, som kan aflæses ved at et bestemt index aflæses. For ikke at være nød til at holde styr 
på flere index's og for at hurtigt kunne aflæse en ny besked er driveren designet til kun at bruge plads 0 i dette array. 
Når en besked er aflæst bliver denne plads i array'et nulstillet, således vil en ny besked altid være på en kendt plads.\\

Information omkring brug af modem'et er indhentet ved hjælp af tilsvarende datasheet og online kilder, se side~\pageref{sec:refs} for referencer.

Selve kommunikationen mellem ATmega32 og MC35 var relativt simpel. Hvorimod behandling of modulering af data fra modemet var en større udfordring.
Kommunikationen og behandling af data er der skrevet en driver, som groft sagt kan deles op i to sektioner. En utility sektion, der har til formål at 
behandle data, samt en command sektion der står for at sende kommandoer fra ATmega32 til MC35 modemet. 

\subsubsection{Polling for Ny SMS}

ATmega32 poller MC35 1 gang hvert sekund, hvor der anmodes om data fra MC35's indikator control. 
Dette gøres for at undgå overheadet der følger med at skulle modtage al dataen fra et SMS index og derefter checke om det er fyldt ud.

Inholdet af indikator control\footnote{MC35 data sheet side 59} giver informtion vedrørende en række status parametre på MC35, herunder:

\begin{itemize}
    \item Battery level indication - Denne bruges ikke i projektet da MC35 er tilsluttet en strømforsyning.
    \item Signal Quality - Spænder fra 0 til 7 eller 99 hvis signalet ikke er målbart (et opkald ikke foretages).
    \item Service availibilty - 0 eller 1. Indikerer om et netværk er tilgængeligt.
    \item Sounder  - ikke relevant.
    \item Unread message indication - 0 eller 1. Sættes til 1 hvis der findes en ulæst SMS i den 
    definerede hukommelse\footnote{SMS data kan gemmes i MC35 egen hukommelse eller på SIM kortet}.
    \item Ongoing call indication - 0 eller 1.
    \item Roaming indication - 0 hvis på eget netværk, 1 hvis på andet.
    \item SMS full indication - ikke relevant da implementeringen af driveren sikrer at en SMS slettes efter den er læst og behandlet.
    \item Signal strength indicaion - 0 til 5 Se databladet s. 60 for opdeling pr. dBm.
\end{itemize}

Når der sendes en forespørgsel på indikator control modtages et svar på følgende format:\\

	CIND?\\
	+CIND: 5,99,1,0,0,0,0,0\\

Der er implementeret en Status struct der kan holde på værdierne af indikator kontrollen, samt en setter funktion der sørger for at læse svaret fra MC35 og derefter sætte værdierne.
I det nuværende system bruges udelukkende indikation for ulæst besked, dog kan dette nemt udvides.

Når programmet på ATmega32 detekterer et '1' på Unread Message, sendes der en anmodning om den specifikke besked. Programmet skiller da den modtagne data ad ved at lææge hhv. meta-data, sender information og kommando 
ud i separate char arrays. Herefter reageres der på indholdet af beskeden, hvorefter der sendes et svar tilbage til afsenderen.

Systemet modtager følgende kommandoer:

\begin{itemize}
    \item 'T' - Anmodning om temperatur data. Bruger får efter få sekunder tilsendt en temperaturmåling i grader Celcius.
    \item 'A' - Anmodning om højde over havoverflading. Brugeren får efter få sekunder tilsendt en beregning af højden.
    \item 'P' - Anmodning om trykdata. Brugeren får efter få sekunder tilsendt en trykmåling i Pascal.
    \item 'C' - Anmodning om liste over kendte kommandoer. Brugeren får efter få sekunder tilsendt en liste over kommandoer.
    \item Ukendte kommandoer - Brugeren får en fejlmeddelselse, samt information om hvordan kendte kommandoer kan tilsendes.
\end{itemize}

En bruger af systemet har mulighed for at få tilsendt alle anerkendte kommandoer til sin mobiltelefon.



% The technical considerations and choices made during the project work.

% Advantages and disadvantages for the possible alternative solutions.

% How the group has acquired the knowledge necessary to do the project work.
