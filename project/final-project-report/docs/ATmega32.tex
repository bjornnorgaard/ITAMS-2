\subsection{ATmega32}

Dette afsnit beskriver brug og programmering af ATmega32 samt begrundelser for valget at netop denne microcontroller.

\subsubsection{Valg af microcontroller}

ATmega32 er en velkendt 8-bits microcontroller fra Atmel, og gruppen har før arbejdet med den i samspil med udviklingsboardet STK500.
ATmega32 understøtter både kommunkation med I2C, SPI og UART/USART og har samtidigt tilstrækkeligt med I/O pins, hvilket gjorde den til det oplagte valg for dette projekt.

I forbindelse med uvikling af software til microcontrolleren, bruges udviklingsværktøjet Atmel Studio 7.0.

\subsubsection{Programmering at ATmega32}

I begyndelsen af udviklingsfasen blev der installeret en bootloader microcontrolleren der 
gjorde self-programming muligt vha. SPM instruktionen. Dette var med henblik på at lave system portabelt 
og uafhængigt af STK500 udviklingsboardet, og/eller evt med henblik på at gøre brug af RWW\footnote{Read While Write} og 
NRWW\footnote{Not Read While Write} memory sektionerne til at udføre kritiske kodesektioner.

% The technical considerations and choices made during the project work.

% Advantages and disadvantages for the possible alternative solutions.

% How the group has acquired the knowledge necessary to do the project work.
